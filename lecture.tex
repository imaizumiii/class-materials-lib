\documentclass{ltjsarticle}

\usepackage{luatexja}
\usepackage[haranoaji, match]{luatexja-preset} % TeX Live 標準の HaranoAji 系を使用

% --- ここがポイント ---
\renewcommand{\familydefault}{\sfdefault}      % 欧文デフォルトをサンセリフに
\renewcommand{\kanjifamilydefault}{\gtdefault} % 和文デフォルトをゴシックに
% ------------------------

\usepackage{amsmath, amssymb, amsthm}
\usepackage{bm}
\usepackage{physics}
\usepackage[margin=12mm]{geometry}
\usepackage[most]{tcolorbox}
\usepackage{xcolor}
\usepackage{anyfontsize}
\AtBeginDocument{\fontsize{11pt}{18pt}\selectfont}

\begin{document}

\begin{tcolorbox}[enhanced, sharp corners=southwest, arc=4pt, colback=white, colframe=gray!30, boxrule=0.4pt, boxsep=6pt, borderline west={3pt}{0pt}{blue!60}]
{\LARGE\bfseries タイトル}
\end{tcolorbox}


\noindent\tcbox[enhanced, colback=white, colframe=white, boxrule=0pt, left=0pt, right=0pt, top=0pt, bottom=2pt, borderline south={0.9pt}{0pt}{blue!60}]{\Large\bfseries セクション}%
\par


\begin{tcolorbox}[enhanced, boxrule=0.4pt, arc=5pt, boxsep=6pt, colback=white, colframe=black, title={{\large\bfseries ●\;タイトル}}, fonttitle=\bfseries, coltitle=black, colbacktitle=blue!6]
add-termsで出力
\end{tcolorbox}


区間 $[a,b]$ を縮めていくと、極限として瞬間変化率が得られます。

\end{document}
