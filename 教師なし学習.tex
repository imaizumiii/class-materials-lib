\documentclass{ltjsarticle}

\usepackage{luatexja}
\usepackage[haranoaji, match]{luatexja-preset} % TeX Live 標準の HaranoAji 系を使用

% --- ここがポイント ---
\renewcommand{\familydefault}{\sfdefault}      % 欧文デフォルトをサンセリフに
\renewcommand{\kanjifamilydefault}{\gtdefault} % 和文デフォルトをゴシックに
% ------------------------

\usepackage{amsmath, amssymb, amsthm}
\usepackage{bm}
\usepackage{physics}
\usepackage[margin=24mm]{geometry}
\usepackage[most]{tcolorbox}
\usepackage{xcolor}
\usepackage{anyfontsize}
\AtBeginDocument{\fontsize{11pt}{18pt}\selectfont}

\begin{document}

\begin{tcolorbox}[enhanced, sharp corners=southwest, arc=4pt, colback=white, colframe=gray!30, boxrule=0.4pt, boxsep=6pt, borderline west={3pt}{0pt}{blue!60}]
{\LARGE\bfseries 教師なし学習}
\end{tcolorbox}


\vspace*{6pt}
\noindent\tcbox[enhanced, colback=white, colframe=white, boxrule=0pt, left=0pt, right=0pt, top=10pt, bottom=2pt, borderline south={0.9pt}{0pt}{blue!60}]{\Large\bfseries k-means法}%
\par
\vspace*{2pt}


\begin{tcolorbox}[enhanced, boxrule=0.4pt, arc=5pt, boxsep=6pt, colback=white, colframe=black, before skip=20pt, after skip=6pt, title={{\large\bfseries ●\;k-means法}}, fonttitle=\bfseries, coltitle=black, colbacktitle=blue!6]
データをk個のクラスタに分ける代表的手法
\end{tcolorbox}


\par\vspace*{0pt}%
\noindent\hspace*{0pt}%
\begin{minipage}{\dimexpr\linewidth - 0pt - 0pt\relax}
\begin{tcbraster}[raster columns=1, raster column skip=6pt, raster left skip=0pt, raster right skip=0pt, raster before skip=0pt, raster after skip=0pt]
\begin{tcolorbox}[enhanced, colback=white, colframe=black, boxrule=0.4pt, arc=4pt, boxsep=0pt, left=0pt, right=0pt, top=0pt, bottom=0pt, height=200pt]\end{tcolorbox}
\end{tcbraster}
\end{minipage}%
\hspace*{0pt}\par\vspace*{10pt}


\vspace*{8pt}\noindent\textbf{▶\;メリット}\\[-24pt]
\begin{itemize}
\setlength{\topsep}{2pt}%
\setlength{\itemsep}{2pt}%
\setlength{\parsep}{0pt}%
\setlength{\partopsep}{0pt}%
  \item シンプルで理解しやすい
  \item 計算が早くスケーラブル
  \item 結果が分かりやすく、可視化しやすい
\end{itemize}\vspace*{4pt}


\vspace*{8pt}\noindent\textbf{▷\;デメリット}\\[-24pt]
\begin{itemize}
\setlength{\topsep}{2pt}%
\setlength{\itemsep}{2pt}%
\setlength{\parsep}{0pt}%
\setlength{\partopsep}{0pt}%
  \item クラスタ数 k を事前に決める必要がある
  \item 外れ値(outlier)に弱い
  \item 特徴量のスケールに敏感
\end{itemize}\vspace*{4pt}


\clearpage


\begin{tcolorbox}[enhanced, sharp corners=southwest, arc=4pt, colback=white, colframe=gray!30, boxrule=0.4pt, boxsep=6pt, borderline west={3pt}{0pt}{blue!60}]
{\LARGE\bfseries 教師なし学習}
\end{tcolorbox}


\vspace*{-6pt}
\noindent\tcbox[enhanced, colback=white, colframe=white, boxrule=0pt, left=0pt, right=0pt, top=10pt, bottom=2pt, borderline south={0.9pt}{0pt}{blue!60}]{\Large\bfseries 階層的クラスタリング}%
\par
\vspace*{-6pt}


\begin{tcolorbox}[enhanced, boxrule=0.4pt, arc=5pt, boxsep=6pt, colback=white, colframe=black, before skip=20pt, after skip=6pt, title={{\large\bfseries ●\;階層的クラスタリング}}, fonttitle=\bfseries, coltitle=black, colbacktitle=blue!6]
データを階層的にまとめ、木構造(デンドログラム)で関係を表す手法
\end{tcolorbox}


\end{document}
